\section{Discussion}

We hope the competition would encourage more people to join in the garbage classification task, proposing better models to handle it. The excellent models can also be applied in real-world garbage classification scenario, reducing the burden of dustmen. We also hope the competition can impress more people, making them paying more attention to garbage classification in daily life.

Due to the time limit, our current work is not perfect enough. There are some limitations to improve in the future. Although about 17,000 images are included in our garbage dataset, the data unbanlanced issue still exists, which makes it harder for our trained models to achieve competitive performance in real-world application as on the dataset. Some public image datasets may be potiential resources for us to extend current garbage dataset.

The baseline evaluation is done at one training using stochastic optimize methods, which is unstable in the final results. A better way to fix that is to train one model multiple times and average the results as the final result, but it may take a massive amount of time.

Garbage usually does not appear as single object, indicating the needs of multiple garbage classification in one image. Integrated with object tracking algorithms, the models would be more powerful and applicable in real-world complex garbage classification tasks. 

\section{Acknowledgement}

In preparation of our project, we had to take the help and guidance of many respected persons, who deserve our deepest gratitude. Firstly sincerely thanks to Professor Xuegong Zhang, who open a window towards machine learning for us. Some basic ideas carried in this course are priceless. As the completion of this assignment gave us much pleasure, we would like to show my gratitude to all the TAs for their thorough answering to our questions and also for recommending good reading materials.


Many people, especially my teammates have made valuable suggestions and conversations along the way. We would like to thank each other's effort.

The division of work:

Yunyue Wei:Baseline performance of popular models, multiple metrics evaluation and visualization

Huahuan Zheng: Data arrangement and visualization of feature maps for interpreting NN model.

Zeji Yi: Advanced model Resnext and its fine-tuning. Data augmentation and visualization heatmap class activation.